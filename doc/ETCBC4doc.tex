\documentclass[11pt,oneside,a4paper,article]{memoir}
\usepackage{fontspec}
\usepackage[unicode=true,xetex,colorlinks=true,linkcolor=blue,urlcolor=blue,bookmarksnumbered=true,bookmarksdepth=3]{hyperref}
\usepackage{bidi} % Must be last

%%%%%%%%%%%%%%%%%%%%%%%%%%%%%%%%%%%%%%%%%%%%%%%%%%%%%%%
%%%%%%%%%%%%%%%%%%% How to typeset %%%%%%%%%%%%%%%%%%%%
%%%%%%%%%%%%%%%%%%%%%%%%%%%%%%%%%%%%%%%%%%%%%%%%%%%%%%%

%% This file must be run through xelatex

%% The following fonts must be accessible to xelatex on
%% your system:
%%
%% Linux Libertine O
%% EzraSIL
%%
%% On Fedora Linux, you can bring in these fonts like
%% so:
%%
%% sudo dnf install sil-ezra-fonts-all linux-libertine-fonts
%%



%%%%%%%%%%%%%%%%%%%%%%%%%%%%%%%%%%%%%%%%%%%%%%%%%%%%%%%
%%%%%%%%%%%%%%%%%%%% Configuration %%%%%%%%%%%%%%%%%%%%
%%%%%%%%%%%%%%%%%%%%%%%%%%%%%%%%%%%%%%%%%%%%%%%%%%%%%%%

%%% Fonts %%%
\setmainfont[Ligatures=TeX]{Linux Libertine O}
\newfontfamily{\ezr}[Script=Hebrew]{EzraSIL}

\newfontfamily{\mainnolig}{Linux Libertine O}
\newcommand{\q}{{\mainnolig '}}

\newcommand{\heb}[1]{{\RL {\ezr #1}}}


%%% Page layout %%%
\settypeblocksize{247mm}{160mm}{*}
\setlrmargins{*}{*}{1}
\setulmargins{*}{*}{1}
\checkandfixthelayout

%%% Hyperref (Information in PDF) %%%
\hypersetup{
unicode=true,
pdfauthor={Claus Tøndering},
pdftitle={Generating ETCBC4}
}

%%% Lists %%%
\tightlists

%%% Allow extra space between words %%%
\sloppy


%%% Font matter %%%
\title{Generating ETCBC4}
\author{Claus Tøndering}
\date{13 January 2023}


\begin{document}
\maketitle


%%%%%%%%%%%%%%%%%%%%%%%%%%%%%%%%%%%%%%%%%%%%%%%%%%%%%%
%%%%%%%%%%%%%%%%%%%% Introduction %%%%%%%%%%%%%%%%%%%%
\chapter{Introduction}

This document gives a brief introduction to how to generate the ETCBC4 Emdros database for Bible
Online Learner on a Linux computer.

%%%%%%%%%%%%%%%%%%%%%%%%%%%%%%%%%%%%%%%%%%%%%%%%%%%%%%%
%%%%%%%%%%%%%%%%%%%% Prerequisites %%%%%%%%%%%%%%%%%%%%
\chapter{Prerequisites}

In order to execute this process, you need to clone the GitHub repository
\texttt{https://github.com/\-EzerIT/ETCBC4BibleOL.git}. In the following description I assume that you
have cloned the repository into a folder called \texttt{ETCBC4BibleOL}.

You also need a version of the original ETCBC4 Emdros database from the Eep Talstra Centre for Bible
and Computer. I use what I believe to be version 1.29 from January 2017.

In the \texttt{ETCBC4BibleOL} you must locate this line in the Makefile:

\vspace{1ex}

\quad\texttt{BHS4=../bhs4/bhs4  \# Location of source Emdros database}

\vspace{1ex}

\noindent
and replace ``\texttt{../bhs4/bhs4}'' with the path name of the original ETCBC4 database.

The \emph{emdros\_updater} program depends on the pcre-cpp package, a
C++ binding for the PCRE regex library. Note that this is PCRE, not
PCRE2.

On Fedora Linux, you can install the required packages with this
command:

\vspace{1ex}

\quad\texttt{sudo dnf install pcre-devel pcre-cpp}

\vspace{1ex}




%%%%%%%%%%%%%%%%%%%%%%%%%%%%%%%%%%%%%%%%%%%%%%%%%%%%%%%%%%%
%%%%%%%%%%%%%%%%%%%% Generating ETCBC4 %%%%%%%%%%%%%%%%%%%%
\chapter{Generating ETCBC4}

Executing the command ``make'' in the \texttt{ETCBC4BibleOL} folder should generate the ETCBC4
Emdros database for Bible OL. This takes slightly more than an hour on my computer.

The steps of the process are detailed in the following sections.

\section{Compiling \emph{emdros\_updater}}

The C++ source code for \emph{emdros\_updater} is compiled. A C++ compiler that supports the
``-std=c++20'' flag is required.

\section{Executing \emph{emdros\_updater}}

\noindent \textbf{Inputs:}
\begin{itemize}
\item The original ETCBC4 database
\item \texttt{ETCBC4-frequency4.04\_progression-heb.csv}
\item \texttt{ETCBC4-frequency4.04\_progression-aram.csv}
\end{itemize}

\noindent \textbf{Output:} \texttt{update.mql}

\vspace{1ex}

The files \texttt{ETCBC4-frequency4.04\_progression-*.csv} are taken from the Excel file
\texttt{ETCBC4-frequency4.04\_progression.xlsx} created by Oliver Glanz. They contain information
for the \emph{fa\_order} and \emph{verb\_class} features.

The \emph{emdros\_updater} program generates data for the following Emdros features:

\begin{itemize}
\item fa\_order
\item frequency\_rank
\item g\_lex\_cons\_utf8
\item g\_lex\_translit
\item g\_nme\_cons\_utf8
\item g\_nme\_translit
\item g\_pfm\_cons\_utf8
\item g\_pfm\_translit
\item g\_prs\_cons\_utf8
\item g\_prs\_translit
\item g\_suffix\_translit
\item g\_uvf\_cons\_utf8
\item g\_uvf\_translit
\item g\_vbe\_cons\_utf8
\item g\_vbe\_translit
\item g\_vbs\_cons\_utf8
\item g\_vbs\_translit
\item g\_voc\_lex\_cons\_utf8
\item g\_voc\_lex\_translit
\item g\_word\_cons\_utf8
\item g\_word\_nocant
\item g\_word\_nocant\_utf8
\item g\_word\_nopunct\_translit
\item g\_word\_nostress
\item g\_word\_nostress\_utf8
\item g\_word\_translit
\item lex\_cons\_utf8
\item lexeme\_occurrences
\item monad\_num
\item qere\_translit
\item verb\_class
\end{itemize}

The execution of \emph{emdros\_updater} takes considerable time, and the progress through the books
of the Bible is shown on the console.

Note that \emph{emdros\_updater} may warn about illegal final characters in a few cases. The warnings
may look like this:

\vspace{1ex}

  \quad\texttt{Bad final character corrected: }\heb{בֵינֶֽיכָ}\texttt{  GEN 16,05}

\vspace{1ex}

In my case I get three of these warnings. They indicate errors in the original ETCBC4 text.

\section{Applying \texttt{update.mql}}

\noindent \textbf{Inputs:}
\begin{itemize}
\item The original ETCBC4 database
\item \texttt{update.mql}
\end{itemize}

\noindent \textbf{Output:} Updated ETCBC4 database

\vspace{1ex}

The ``mql'' command is executed and updates ETCBC4 with the information generated in the previous
step. During this process the ``mql'' command outputs more than two million lines of text to the
console.

\section{Executing \emph{change\_mql.sh}}

\noindent \textbf{Input:} The ETCBC4 database

\noindent \textbf{Output:} Updated ETCBC4 database

\vspace{1ex}

The program ``mqldump'' is executed on the ETCBC4 database, the output is piped through the shell
script \emph{change\_mql.sh} which adds ``FROM SET'' and ``WITH INDEX'' to a number of features.

Finally, the ETCBC4 database is recreated from the updated MQL file.

\section{Executing \emph{fix\_lower\_dots.sh}}

\noindent \textbf{Input:} The ETCBC4 database

\noindent \textbf{Output:} Updated ETCBC4 database

\vspace{1ex}

This shell script \emph{fix\_lower\_dots.sh} is executed. It changes the lower dots of the first
word of Ps 27:13 from Unicode character U+0323 to U+05C5.


\section{Executing \emph{extra\_tenses/make\_update.sh}}

\noindent \textbf{Inputs:}
\begin{itemize}
\item The ETCBC4 database
\item \texttt{extra\_tenses/cohortative.mql}
\item \texttt{extra\_tenses/emphatic\_imperative.mql}
\item \texttt{extra\_tenses/jussive.mql}
\end{itemize}

\noindent \textbf{Output:} Updated ETCBC4 database

\vspace{1ex}

The shell script \emph{make\_update.sh} in the folder \emph{extra\_tenses} is executed. It uses
three MQL files in that folder to identify certain verbs and adds the extra verbal tenses
\emph{juss, coho,} and \emph{emin} to the verbs. The MQL files were created by Oliver Glanz.

\vspace{1ex}

\noindent
At this point the ETCBC4 database is finished.

\section{Compiling \emph{worddb}}

The C++ source code for \emph{worddb} is compiled.


\section{Executing \emph{worddb}}

\noindent \textbf{Input:} The ETCBC4 database.

\noindent \textbf{Output:} The \texttt{ETCBC4\_words.db} database.

\vspace{1ex}

The \emph{worddb} program is executed. It creates the so-called ``words database''. For information
about this database, see the chapter \emph{Multiple-Choice Questions} in the Bible OL technical
documentation.

\section{Compiling \emph{hintsdb}}

The C++ source code for \emph{hintsdb} is compiled.


\section{Executing \emph{hintsdb}}

\noindent \textbf{Input:} The ETCBC4 database.

\noindent \textbf{Output:} The \texttt{ETCBC4\_hints.db} database.

\vspace{1ex}

The \emph{hintsdb} program is executed. I creates the so-called ``hints database''. For information
about this database, see the chapter \emph{Hints} in the Bible OL technical
documentation.


%%%%%%%%%%%%%%%%%%%%%%%%%%%%%%%%%%%%%%%%%%%%%%%%%%%%%%%%%%%%
%%%%%%%%%%%%%%%%%%%% Generating Lexicons %%%%%%%%%%%%%%%%%%%%
\chapter{Generating Lexicons}

There are currently programs available for generating these lexicons:
\begin{itemize}
\item Aramaic-English
\item Hebrew-Spanish
\end{itemize}

Executing the command ``make build\_aram\_lex\_file'' in the \texttt{ETCBC4BibleOL} folder compiles the
program \emph{build\_aram\_lex\_file}. The program is then executed. It takes input from
\texttt{ETCBC4-frequency4.02\_progression-aram.csv} (provided by Oliver Glanz) and generates
\texttt{aram\_en.csv} which can be uploaded to Bible OL as an Aramaic-English dictionary.

Similarly, executing the command ``make build\_heb\_es\_lex\_file'' in the \texttt{ETCBC4BibleOL} folder compiles the
program \emph{build\_heb\_es\_lex\_file}. The program is then executed. It takes input from
\texttt{BibleOL\_dictionary\_Hebrew-Spanish\_v1.2.csv} (provided by Oliver Glanz) and generates
\texttt{heb\_es.csv} which can be uploaded to Bible OL as a Hebrew-Spanish dictionary.

These two programs can be used as templates for creating other lexicons.


\end{document}

% Local Variables:
% mode: latex
% ispell-dictionary: "british-ize"
% ispell-extra-args: ("--home-dir=/home/claus/Projects/BibleOL/techdoc")
% eval: (auto-fill-mode 1)
% End:
